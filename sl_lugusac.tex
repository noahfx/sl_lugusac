\documentclass{beamer}
\usepackage[spanish]{babel}
\parskip 10.9pt


\usetheme{CambridgeUS}
\title{¿Qu\'e es el Software Libre?}
\author[noahfx]{Josu\'e Ortega \\ \texttt{http://openfecks.wordpress.com}}
\institute{SSL 20 - Lugusac}
\begin{document}

\begin{frame}
	\titlepage
\end{frame}

\begin{frame}{Esquema}
	\tableofcontents
\end{frame}

\section{Introducci\'on}
\subsection{Software Libre}
\begin{frame}
	  \centering
	  \alert ¿Qu\'e es Software Libre?
	  \pause
	   el software libre se refiere a la libertad de los usuarios para ejecutar, copiar, distribuir, 
	    estudiar, modificar el software y distribuirlo modificado. 
\end{frame}


\section{Historia}

\subsection{A\~nos 70}
\begin{frame}{Historia}
  \centering
     \alert {\bf A\~nos 70}
      \begin{itemize}
	\item En los a\~nos 70 el software se utilizaba en \'ambitos universitarios y empresariales
	\pause
	\item Los programadores creaban y compart\'ian el software sin ningun tipo de restricciones
      \end{itemize}
 \end{frame}


\subsection{A\~nos 80}
\begin{frame}{Historia}
	  \centering	 
	  \alert {\bf A\~nos 80}
	  \begin{itemize}
		\item Richard Stallman crea el proyecto GNU (GNU is not Unix) (1983)
		\pause
		\item Free Software Foundation (1985)		
	  \end{itemize}
\end{frame}
\subsection{A\'nos 90}
\begin{frame}{Historia}
	  \centering
	   \alert  {\bf A\~nos 90} 
	   \begin{itemize}
	      \item Kernel Linux desarrollado por Linus Torvalds
	      \item Nacimiento de GNU Linux
              \item {\bf Multiples distribuciones:}  Debian GNU/Linux, Ubuntu, OpenSuse, ArchLinux
	   \end {itemize}
\end{frame}

\section{Libertades}
\begin{frame}{Libertades}
\centering
       \alert {\bf Libertades }\\
      El {\em software libre} es una cuestion de {\em libertad}, no de precio. 
      Para entender el concepto se debe pensar en {\em libre} como en {\em libre expresion}, no como en {\em barra libre}
\end{frame}

\subsection{Libertad 0}
\begin{frame}{Libertades}
      \centering
       \alert {\bf Libertad 0} \\
	La libertad de usar el programa, con cualquier prop\'osito.
  
\end{frame}
\subsection{Libertad 1}
\begin{frame}{Libertades}
      \centering
       \alert {\bf Libertad 1} \\
	La libertad de estudiar c\'omo funciona el programa y modificarlo, adapt\'andolo a tus necesidades.
\end{frame}

\subsection{Libertad 2}
\begin{frame}{Libertades}
      \centering
      \alert {\bf Libertad 2} \\
      La libertad de distribuir copias del programa, con lo cual puedes ayudar a tu pr\'ojimo.
\end{frame}

\subsection{Libertad 3}  
\begin{frame}{Libertades}
  \centering 
    \alert {\bf Libertad 3} \\
      La libertad de mejorar el programa y hacer p\'ublicas esas mejoras a los dem\'as, de modo que toda la comunidad se beneficie.\\
      \pause
      \alert {Software de Mayor Calidad}
\end{frame}

\section{Licencias}

 \subsection{Copyleft}
  \begin{frame}{Licencias}
     \begin{itemize}
	\item Qu\'e pasa si alguien se apropia del c\'odigo?
	 \pause
	 \item {\bf Copyleft:} Se practica al ejercer el derecho de autor que consiste en permitir la libre distribuci\'on de copias 
		y versiones modificadas de una obra u otro trabajo, exigiendo que los mismos derechos sean preservados en las versiones 
		modificadas.
	\pause
	\item GPL Licencia P\'ublica General de GNU (GNU General Public Licence)
      \end{itemize}
  \end{frame}
\section{Qu\'e no es Software Libre?}
  \begin{frame}{Qu\'e no es Software Libre?}
   \begin{itemize}
      \item Freeware
      \pause
      \item Programas que permiten ver su c\'odigo fuente pero no se puede modificar ni redistrubuir
      \pause
      \item Cualquier pieza de software que no cumpla con las 4 libertades del Software Libre.
      \pause
   \end{itemize}
  \end{frame}

\section{D\'onde puedo conseguir Software Libre?}
  \begin{frame}
    \begin{itemize}
      \item Aunque no sea un requisito la mayor\'ia de veces se puede encontrar software libre dede forma gratuita.
      \pause
      \item Internet, amigos que usen software libre
      \item Comunidades Locales
      \end{itemize}
   \end{frame}

\section {Comunidades en Guatemala}
\begin{frame}
  \begin{itemize}
      \item Software Libre Guatemala (SLGT) http://slgt.org/ 
      \item Grupo de usuarios de GNU/Linux de la Universidad de San Carlos http://lugusac.org/
    \end{itemize}
   \end{frame}

\section {Preguntas????}
\begin{frame}
      \huge Preguntas? \\
      \pause
       \alert { Gracias :)}

	  {\tiny
    \begin{center}
      \begin{tabular}{l@{\hspace{1em}}l}
       
        
       licencia
        & \href{http://creativecommons.org/licenses/by-sa/3.0/}{CC BY-SA 3.0 ---
          Creative Commons Attribution-ShareAlike 3.0} \\
       
      \end{tabular}
    \end{center}}
\end{frame}

\end{document}
